\begin{titlepage}
	
\thispagestyle{empty}

\begin{minipage}[t]{0.99\textwidth}
		\centering
		\includegraphics[width=0.98\textwidth]{figures/TUlogo.pdf}
\end{minipage}

Project gruppe f�r \autor\\
\textbf{\titel}

\hrule
\vspace{0.2cm}

Die Integration der regenerativen Erzeugungseinheiten in den europ�ischen Kraftwerkspark und damit in das europ�ische Verbundnetz stellt derzeit eine der gr��ten Herausforderungen in der �kologischen Modernisierung der elektrischen Energieversorgung dar. Innerhalb der regenerativen Energiequellen wird derzeit der Windkraft das gr��te Potential zur Ersetzung
gr��erer, konventioneller Erzeugungsbl�cke zugesprochen. Ihre Standorte befinden sich zuk�nftig jedoch dargebotsbedingt zumeist auf hoher See (sogenannte Offshore"=Windparks) und damit weit entfernt von den nationalen �bertragungsnetzen, so dass ihr Netzanschluss h�ufig nur mittels der Hochspannungsgleichstrom�bertragung (HG�) realisiert werden kann.
Aus Netzsicht erscheint daher nicht mehr der rotierende, massebehaftete Generator als Einspeiser, sondern lediglich der leistungselektronische, tr�gheitslose Umrichter. Im Falle der �berwiegenden Einspeisung aus leistungselektronischen Umrichtern bedeutet dies f�r ein Netz, in dem zur Herstellung des Leistungsgleichgewichts eine Frequenzregelung eingesetzt wird, dass die Leistungsf�higkeit und die Grenzen dieses Regelalgorithmus neu bewertet werden m�ssen. In diesem Zusammenhang stellt sich auch die Frage, inwiefern HG�"=Umrichter einen Beitrag zur Bereitstellung von Momentanreserve und zur Prim�rregelung leisten k�nnen.

Folgende Strukturierung der Arbeit wird vorgeschlagen:

\begin{itemize}
	\item Einarbeitung in die Grundlagen der Hochspannungsgleichstrom�bertragung, Leistungs"~/""Frequenzregelung und Simulationsumgebung PowerFactory
	\item Ermittlung des Einflusses leistungselektronischer Einspeisung auf die Momentanreserve und die Leistungsf�higkeit einer Leistungs"~/""Frequenzregelung in einem Drehstromnetz
	\item Entwurf einer Umrichterregelung, die die Beteiligung des Umrichters an der Bereitstellung der Momentanreserve und der Prim�rregelung erm�glicht
	\item Absch�tzung der Leistungsf�higkeit einer solchen Umrichterbeteiligung
	\item Validierung der gefundenen Zusammenh�nge und Ergebnisse an einem Beispielnetz
	\item Dokumentation der durchgef�hrten Arbeiten
\end{itemize}

Im Anschluss an diese Arbeit ist in einem Vortrag �ber die Ergebnisse zu berichten.

\begin{tabbing}
		Tag der Ausgabe:	\hspace{3.5cm} \= \ausgabetag\\
		Tag der Abgabe: 	\> 								\abgabetag\\
		Zust�ndig: 				\> 								\betreuer\\
											\>								\cobetreuer
\end{tabbing}	
	
\end{titlepage}